% This must be in the first 5 lines to tell arXiv to use pdfLaTeX, which is strongly recommended.
\pdfoutput=1
% In particular, the hyperref package requires pdfLaTeX in order to break URLs across lines.

\documentclass[11pt]{article}

% Remove the "review" option to generate the final version.
\usepackage{acl}

% Standard package includes
\usepackage{times}
\usepackage{latexsym}

% For proper rendering and hyphenation of words containing Latin characters (including in bib files)
\usepackage[T1]{fontenc}

% This assumes your files are encoded as UTF8
\usepackage[utf8]{inputenc}

% This is not strictly necessary, and may be commented out,
% but it will improve the layout of the manuscript,
% and will typically save some space.
\usepackage{microtype}

\title{Proposal: Knowledge Graphs for Explainable Reinforcement Learning}

\author{Roel Leenders \\
  University of Twente, Enschede, The Netherlands \\
  \texttt{r.leenders@student.utwente.nl}} 
\begin{document}
\maketitle
% --------------------------------------------
% Rel4KC: A Reinforcement Learning Agent for Knowledge Graph Completion and Validation: http://www.cse.msu.edu/~zhaoxi35/DRL4KDD/1.pdf
% Mining Implicit Entity Preference from User-Item Interaction Data for Knowledge Graph Completion via Adversarial Learning: https://arxiv.org/abs/2003.12718
% Learning to Update Knowledge Graphs by Reading News: https://aclanthology.org/D19-1265.pdf
% Building Dynamic Knowledge Graphs from Text-based Games: https://grlearning.github.io/papers/80.pdf
% Knowledge Infused Learning (K-IL): Towards Deep Incorporation of Knowledge in Deep Learning: https://arxiv.org/abs/1912.00512
% Towards mental time travel: a hierarchical memory for reinforcement learning agents: https://arxiv.org/abs/2105.14039#:~:text=Reinforcement%20learning%20agents%20often%20forget,is%20followed%20by%20distractor%20tasks.

% Knowledge graphs as tools for explainable machine learning: A survey: https://www.sciencedirect.com/science/article/pii/S0004370221001788
% Explainable Reinforcement Learning: A Survey: https://arxiv.org/pdf/2005.06247.pdf
% Playing a Strategy Game with Knowledge-Based Reinforcement Learning: https://arxiv.org/pdf/1908.05472.pdf
% ENVIRONMENTS: https://agents.inf.ed.ac.uk/blog/multiagent-learning-environments/
% Keywords: knowledge graph completion (KGC),
\section{Introduction}
% ----------------------------------------
% The introduction gives the context of the work and also the motivation for research in this area. Meant 
% is the „external‟ motivation: what is the relevance of this subject.
% ----------------------------------------

% <something about why XAI is important>

% <something about why Knowledge base explanation systems (KBX-systems)>
% <something about how KBX-systems haven't really  but not in reinforcement learning>
% <something about that knowledge graphs have been used to explain large language models>
% <something about different XAI methods that have been used within reinforcement learning>

% - Original problem I'd like to fix: Updating knowledge graphs of dynamic environments based on an agent's observations
% - Research Question: How can you update an agent's knowledge of a dynamic environment based on its observations?
% - Context of the work: 
%   - Important for interactions between systems that share info through observations. 
%   - May improve explainability of deep reinforcment learning models
%   - May improve quality of deep reinforcement learning models
% - Relevance of the paper (external motivation)

\section{Problem Statement}
% ----------------------------------------
% This component describes the problem the paper will focus on. The motivation of its relevance and its 
% relationship with the state of the art in the scientific field should be given, with references to the 
% literature. This leads to a number of clearly formulated research questions (usually 3 to 6). The research  
% questions should be operational: they can be answered by the method you are using. Expected answers 
% can be given.  
% A hierarchy in the research questions (main questions with sub questions) can be suitable. In this case 
% the relationship between the main question and the sub questions should be clear. 
% Preferably the research questions are emphasized in lay-out. 
% An important aspect of this component is that it should make clear what the paper contributes to the 
% topic: what is new, what does the research add to the existing knowledge. 
% ----------------------------------------

% <Problem: Currently deep reinforcement learning models are difficult to explain. Knowledge Graphs are an option for better XAI.
% They, however, have not fully been researched for reinforcement learning models>
% <Research Question 1: To what extent, if any, can a knowledge-graph improve the explainability of a deep RL model?>
% <Sub Question 1: How to construct a knowledge graph of a deep RL model?>

\cite{Arnold2013EI}


\section{Proposed Method of Research}
% ----------------------------------------
% This component describes the methodology you are going to use to answer the research questions and 
% motivates why it is suitable. 
% A remark such as „I will perform a literature study to find answers to the questions‟ is too vague. What 
% type of literature, how do you think the literature will lead you to the answers, etc, are all relevant 
% issues. If your methodology involves experiments, you should show how you intend to set up the 
% experiment, how you think the experiment will answer your questions, etc. 
% ----------------------------------------

\section{Planning}
% ----------------------------------------
% Mile-stones in terms of concrete content/tasks/products to be delivered should be given. Elements of 
% the schedule should relate to steps in your methodology. E.g., if you need to find something from the 
% literature before you can set up your experiment, write down when you need to have it. Say when you 
% will start the experiment. Say when you will start collecting data. Say how long you will take for 
% analysis. Say when you expect to have the answers to your questions.  
% NB Tasks such as “week 45: research” are NOT concrete enough. 
% ----------------------------------------
\subsection{References}

\bibliography{anthology,custom}

\appendix

\section{Example Appendix}
\label{sec:appendix}

This is an appendix.

\end{document}
